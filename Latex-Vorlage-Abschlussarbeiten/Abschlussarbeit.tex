%%%%%%%%%%%%%%%%%%%%%%%%%%%%%%%%%%%%%%%%%%%%%%%%%%%%%%%%%%%
% Vorlage f. Abschlussarbeiten														%
%	Hochschule Bochum, it:matters														%
% MIRo-Lab																								%
%	Autor:		Dr. Sven Seiler <sven.seiler@hs-bochum.de>		%
%																													%
%Fr�here Autoren:																					%
%						Martin B�lter	<martin.boelter@hs-bochum.de>		%
%						Jens Jakobi	<jens.jakobi@hs-bochum.de>				%
%																													%
%																													%
% Version: 0.91, 07-02-2013																%
%%%%%%%%%%%%%%%%%%%%%%%%%%%%%%%%%%%%%%%%%%%%%%%%%%%%%%%%%%%

%
% F�r diese Vorlage �bernehmen die Autoren keine Gew�hr.
% Bitte kl�ren Sie mit Ihren Pr�fern ab, ob und in welcher
% Form Sie diese Vorlage f�r Ihre Arbeit verwenden d�rfen.
\newcommand{\DeinName}{Anthony Mendil}
\newcommand{\TitelArbeit}{Using real and simulated bevavioural data to determine whether players were blinded during a game of memory}
\newcommand{\Labor}{Cognitive Systems Lab (CSL)}
\newcommand{\PrueferEins}{Dr. Felix Putze}
\newcommand{\PrueferZwei}{Unknown}
\newcommand{\Supervisor}{Mazen Salous}
\newcommand{\fachbereich}{Mathematics and Computer Science}
% entweder ein Datum h�ndisch eintragen, oder den Befehl \today nutzen
\newcommand{\Datum}{\today}

\makeglossary


%
% Einbinden des Headers, hier k�nnen auch weitere Einstellungen vorgenommen werden.
\input{source/header}


% Beginn des Dokuments
\begin{document}


%Zitiert alle Referenzen, ohne Sie hier zu listen. Dadurch erscheinen alle Quellen im %Literaturverzeichnis, auch wenn sie im Text nicht genutzt werden.
%Bitte nur f�r Testzwecke verwenden.
\nocite{*}


% Einbinden des Deckblatts

\frontmatter
% erste Seite (Titelseite) der Diplomarbeit mit Titel, Name usw...

\newsavebox{\Prof}
\savebox{\Prof}{Erster Pr�fer }

\newsavebox{\Betr}
\savebox{\Betr}{Zweiter Pr�fer }

\newsavebox{\Examiner}
\savebox{\Examiner}{}



\begin{titlepage}
%\begin{textblock}{50}(150,30)
  %\includegraphics[height=3cm]{source/images/BO-Logo_m_Wortmarke_P10cmWebHQ}
	%\includegraphics[height=1cm]{source/images/BO-Logo_m_Wortmarke_L10cmWebHQ}	
%	\includegraphics[height=4cm]{source/images/BO-Logo_m_Wortmarke_P10cmWebHQ}
%\end{textblock}
\begin{center}
%\includegraphics[height=2cm]{source/images/logo_mirolab.png}

{\large University of Bremen}\\[1mm]
{\small Faculty: \fachbereich}\\
{\small Computer Science Bachelor}
\vspace{1.5cm}

{\huge \TitelArbeit}

\vspace{1,5cm}

{\large Bachelor Thesis}

\vspace{1cm}

by\\[2mm]

\textbf{\large{\DeinName}}\\

\vspace{1cm}
\Labor\\[1cm]

Examiner: \\[2mm]

\usebox{\Examiner \PrueferEins} \\
\usebox{\Examiner \PrueferZwei} \\

\vspace{1cm}

Supervisor: \\[2mm]

\usebox{\Examiner \Supervisor} \\

\vspace{1cm}
Bremen, date 
\end{center}
\begin{textblock}{50}(30,250)
  %\includegraphics[height=3cm]{source/images/BO-Logo_m_Wortmarke_P10cmWebHQ}
	\includegraphics[height=1cm]{source/images/uni_logo}	
	
\end{textblock}

%% Hier kann nat�rlich auch jedes andere Logo eingebunden werden! %%
\begin{textblock}{40}(142,248)
  \includegraphics[height=1.5cm]{source/images/csl_logo}
\end{textblock}
\end{titlepage}



\include{source/content/EidesstattlicheErklaerung}
%\includepdf[noautoscale=true]{source/content/Deckblatt_der_Diplomarbeit0901.pdf}

% Der Abstract, bitte in englischer Sprache verfassen!
\chapter{Abstract}
\label{Abstract}
Der Abstract sollte auf Englisch verfasst werden. Er ist eine komprimierte Wiedergabe der wesentlichen Erkentnisse die w�hrend der Abschlussarbeitsphase gewonnen wurden. Der Abstract soll dem Leser eine Entscheidungsgrundlage liefern, ob der Text f�r ihn/sie interessant und lesenswert ist.
\\
\todo[inline]{Der Abstract sollte zum Ende der Arbeit geschrieben werden...}
Lorem ipsum dolor sit amet, consetetur sadipscing elitr, sed diam nonumy eirmod tempor invidunt ut labore et dolore magna aliquyam erat, sed diam voluptua. At vero eos et accusam et justo duo dolores et ea rebum. Stet clita kasd gubergren, no sea takimata sanctus est Lorem ipsum dolor sit amet. Lorem ipsum dolor sit amet, consetetur sadipscing elitr, sed diam nonumy eirmod tempor invidunt ut labore et dolore magna aliquyam erat, sed diam voluptua. At vero eos et accusam et justo duo dolores et ea rebum. Stet clita kasd gubergren, no sea takimata sanctus est Lorem ipsum dolor sit amet. Lorem ipsum dolor sit amet, consetetur sadipscing elitr, sed diam nonumy eirmod tempor invidunt ut labore et dolore magna aliquyam erat, sed diam voluptua. At vero eos et accusam et justo duo dolores et ea rebum. Stet clita kasd gubergren, no sea takimata sanctus est Lorem ipsum dolor sit amet. 
Lorem ipsum dolor sit amet, consetetur sadipscing elitr, sed diam nonumy eirmod tempor invidunt ut labore et dolore magna aliquyam erat, sed diam voluptua. At vero eos et accusam et justo duo dolores et ea rebum. Stet clita kasd gubergren, no sea takimata sanctus est Lorem ipsum dolor sit amet. Lorem ipsum dolor sit amet,  


% Einbinden des Inhaltsverzeichnisses
%\todo[inline]{Inhaltsverzeichnis-Schachtelungswerte auf Standard zur�cksetzen!}
%\setcounter{secnumdepth}{4} % Schachtelungstiefe der Nummerierung von �berschriften
%\setcounter{tocdepth}{4} % Schachtelungstiefe des Inhaltsverzeichnisses
\tableofcontents 

%Einbinden des Hauptteils
\mainmatter

\begin{table}
	\centering
	\caption{2D CNN. 20 turns. config2}%\label{tab1}
	\begin{tabular}{|l|l|l|}
		\hline
		Simulated Games & Best Accuracy (Epoch) & Best Loss (Epoch)\\
		\hline
		0 & 0.7888 (14) & 0.4822 (51) \\
		1 &  &  \\
		2 &  &  \\
		3 &  &  \\
		4 &  &  \\
		5 & 0.8450 (17) & 0.4768 (20) \\
		6 &  &  \\
		7 &  &  \\
		8 &  &  \\
		9 &  &  \\
		10 & 0.8475 (20) & 0.4796 (10) \\
		20 & 0.8437 (6) & 0.4763 (6) \\
		\hline
	\end{tabular}
\end{table}

\begin{table}
	\centering
	\caption{2D CNN. 20 turns. config1}%\label{tab1}
	\begin{tabular}{|l|l|l|}
		\hline
		Simulated Games & Best Accuracy (Epoch) & Best Loss (Epoch)\\
		\hline
		0 & 0.7963 (18) & 0.4931 (98) \\
		1 &  &  \\
		2 &  &  \\
		3 &  &  \\
		4 &  &  \\
		5 & 0.8437 (53) & 0.4821 (31) \\
		6 &  &  \\
		7 &  &  \\
		8 &  &  \\
		9 &  &  \\
		10 &  &  \\
		20 &  &  \\
		\hline
	\end{tabular}
\end{table}



\chapter{Einleitung}
\section{Motivation}
Hier sollten die folgenden Fragestellungen bearbeitet werden:

\begin{itemize}
	\item Warum soll das Thema bearbeitet werden, was war der ausschlaggebende Grund dazu?
	\item Gibt es vll. geleistete Vorarbeit aus Labort�tigkeit, oder wurde das Thema von einem Betreuuer vorgeschlagen?
	\item In welchem Kontext steht das Thema zur akademische Ausbildung?
	\item Warum ist es sinnvoll das Thema zu bearbeiten?
\end{itemize}

\section{Aufgabenstellung}
Wie lautet die Aufgabenstellung?
\todo[inline]{Hier muss noch die Aufgabenstellung erg�nzt werden!}

Lorem ipsum dolor sit amet, consetetur sadipscing elitr, sed diam nonumy eirmod tempor invidunt ut labore et dolore magna aliquyam erat, sed diam voluptua. At vero eos et accusam et justo duo dolores et ea rebum. Stet clita kasd gubergren, no sea takimata sanctus est Lorem ipsum dolor sit amet. Lorem ipsum dolor sit amet, consetetur sadipscing elitr, sed diam nonumy eirmod tempor invidunt ut labore et dolore magna aliquyam erat, sed diam voluptua. At vero eos et accusam et justo duo dolores et ea rebum. Stet clita kasd gubergren, no sea takimata sanctus est Lorem ipsum dolor sit amet. Lorem ipsum dolor sit amet, consetetur sadipscing elitr, sed diam nonumy eirmod tempor invidunt ut labore et dolore magna aliquyam erat, sed diam voluptua. At vero eos et accusam et justo duo dolores et ea rebum. Stet clita kasd gubergren, no sea takimata sanctus est Lorem ipsum dolor sit amet. 
Lorem ipsum dolor sit amet, consetetur sadipscing elitr, sed diam nonumy eirmod tempor invidunt ut labore et dolore magna aliquyam erat, sed diam voluptua. At vero eos et accusam et justo duo dolores et ea rebum. Stet clita kasd gubergren, no sea takimata sanctus est Lorem ipsum dolor sit amet. Lorem ipsum dolor sit amet, consetetur sadipscing elitr, sed diam nonumy eirmod tempor invidunt ut labore et dolore magna aliquyam erat, sed diam voluptua. At vero eos et accusam et justo duo dolores et ea rebum. Stet clita kasd gubergren, no sea takimata sanctus est Lorem ipsum dolor sit amet. Lorem ipsum dolor sit amet, consetetur sadipscing elitr, sed diam nonumy eirmod tempor invidunt ut labore et dolore magna aliquyam erat, sed diam voluptua. At vero eos et accusam et justo duo dolores et ea rebum. Stet clita kasd gubergren, no sea takimata sanctus est Lorem ipsum dolor sit amet. 
Lorem ipsum dolor sit amet, consetetur sadipscing elitr, sed diam nonumy eirmod tempor invidunt ut labore et dolore magna aliquyam erat, sed diam voluptua. At vero eos et accusam et justo duo dolores et ea rebum. Stet clita kasd gubergren, no sea takimata sanctus est Lorem ipsum dolor sit amet. Lorem ipsum dolor sit amet, consetetur sadipscing elitr, sed diam nonumy eirmod tempor invidunt ut labore et dolore magna aliquyam erat, sed diam voluptua. At vero eos et accusam et justo duo dolores et ea rebum. Stet clita kasd gubergren, no sea takimata sanctus est Lorem ipsum dolor sit amet. Lorem ipsum dolor sit amet, consetetur sadipscing elitr, sed diam nonumy eirmod tempor invidunt ut labore et dolore magna aliquyam erat, sed diam voluptua. At vero eos et accusam et justo duo dolores et ea rebum. Stet clita kasd gubergren, no sea takimata sanctus est Lorem ipsum dolor sit amet. 

\section{Aspekte die ich aufgreifen will}
- hi


\include{source/content/Grundlagen}
%\include{source/content/Test}

%Das Fazit
\include{source/content/Fazit}
%Einbinden des Abbildungsverzeichnisses

\backmatter
%Liste der Tabellen
\listoftables
%Einbinden des Tabellenverzeichnisses
\listoffigures
%Einbinden des Sourcecodeverzeichnisses
\lstlistoflistings

% Quellenverzeichnis
\bibliographystyle{plain}
\bibliography{source/bib/references}

% Anhang
\appendix
\include{source/content/Danksagung}
\end{document}



%
% EOF
%