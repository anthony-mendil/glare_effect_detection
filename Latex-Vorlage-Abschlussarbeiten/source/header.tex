%%%%%%%%%%%%%%%%%%%%%%%%%%%%%%%%%%%%%%%%%%%%%%%%%%%%%%%
%																					%
%	In dieser Datei werden alle Packages eingebunden, 	%
% welche f�r das Dokument n�tig sind. Desweiteren 		%
% werden die Dokumentinformationen gesetzt.						%
%																											%
%%%%%%%%%%%%%%%%%%%%%%%%%%%%%%%%%%%%%%%%%%%%%%%%%%%%%%%
%
%	Die KOMAScript Dokumentklasse "scrbook" verwenden.
%
\documentclass[pdftex, 		%
							a4paper, 		% DIN A4 verwenden
							titlepage,	% separate Titelseite
							%draft,			%	Draft-Version, keine Bilder im pdf!
							final,			% Final-Version
							oneside,		% einseitiger Druck
							11pt,				% Schriftgr��e 12pt
							DIV=calc,
							%tocbasic,
							]{scrbook}	%	KOMAScript scrbook-Dokumentklasse
							
%%%%%%%%%%%%%%%%%%%%%%%%%%%%%%%%%%%%%%%%%%%%%%%%%%%%%%%%
%	Einbinden der Pakete 
%%%%%%%%%%%%%%%%%%%%%%%%%%%%%%%%%%%%%%%%%%%%%%%%%%%%%%%%

\usepackage[ngerman]{babel}

% PDF Dateien einbinden
\usepackage{pdfpages}

%Settings for PDF Pages to accept additonal versioned PDF files
\pdfminorversion=6
\pdfcompresslevel=9
\pdfobjcompresslevel=9

%Infos dazu unter: http://www.bakoma-tex.com/doc/latex/koma-script/scrhack.pdf
%Einige Pakete haben Probleme mit dem Komaskript.
\usepackage{scrhack} 


% Definieren von eigenen benannten Farben.
% F�r sp�tere Verwendung in dem Dokument, definieren wir einzelne
% benannte Farben.
%
\usepackage{xcolor}
\definecolor{gray1}{gray}{0.92}
\definecolor{darkgreen}{rgb}{0,0.5,0}

\definecolor{urlLinkColor}{rgb}{0,0,0.5}
\definecolor{LinkColor}{rgb}{0,0,0}
\definecolor{ListingBackground}{rgb}{0.85,0.85,0.85}




\rmfamily
\usepackage{natbib}
\usepackage{palatino} 				% Schriftfamilie Palatino
\usepackage[latin1]{inputenc} % Umlaute  
%\usepackage[dvips]{color}    	% f�r graue Boxen
%\usepackage[dvips]{graphicx} 	% Grafikpaket
\usepackage{makeidx}   				% Paket zur Erzeugung eines Index
\usepackage[normalem]{ulem}   % bietet Unterstreichungsvarianten
\usepackage{picins} 					% Bilder im Absatz platzieren
\usepackage[T1]{fontenc}			% Erweiterten Zeichensatz aktivieren
\usepackage{multido}					% erm�glicht Schleifenartiges wiederholen von Befehlen
\usepackage{mdwlist}					% erm�glicht das Setzen des Z�hlers bei Aufz�hlungspunkten
\usepackage{paralist}					% Paket f�r Aufz�hlungen, erweitert Enumerate-Paket
\usepackage{longtable}				% mehrseitige Tabellen
\usepackage{tocbasic}
\parindent0pt           			% verzichte auf Einr�cken der ersten Zeile
\parskip1ex             			% Abstand zwischen den Abs�tzen

\usepackage{setspace}					% Paket zum Einstellen des Zeilenabstands
\onehalfspacing								% anderthalbfacher Zeilenabstand
%\doublespacing								% doppelter Zeilenabstand
%\singlespacing								% einfacher Zeilenabstand

%\usepackage[german]{babel}
\usepackage[german=quotes]{csquotes} %Deutsche Anf�hrungszeichen

\usepackage{color}
\definecolor{LinkColor}{rgb}{0.1,0.1,0.1}
%\definecolor{ListingBackground}{rgb}{0.85,0.85,0.85}
\definecolor{ListingBackground}{rgb}{0.98,0.98,0.98}
\definecolor{gray}{rgb}{0.4,0.4,0.4}
\definecolor{darkblue}{rgb}{0.0,0.0,0.6}
\definecolor{cyan}{rgb}{0.0,0.6,0.6}


%
% Farbeinstellungen f�r die Links im PDF Dokument.
%
\makeindex

%-----------Paket f�r absolute Positionierung von Grafiken------------------
\usepackage[absolute]{textpos}
\setlength{\TPHorizModule}{1mm}
\setlength{\TPVertModule}{\TPHorizModule}

%-----------Aufz�hlungen und Einstellungen f�r Sourcecode-------------------
%\usepackage[savemem]{listings} %Bei wenig Arbeitsspeicher dies Option [savemem] aktivieren.
\usepackage{listings}
\lstloadlanguages{TeX,XML, Java} % TeX sprache laden, notwendig wegen option 'savemem'
\lstset{%
	language=[LaTeX]TeX,     % Sprache des Quellcodes ist TeX
	numbers=left,            % Zelennummern links
	stepnumber=1,            % Jede Zeile nummerieren.
	numbersep=5pt,           % 5pt Abstand zum Quellcode
	numberstyle=\tiny,       % Zeichengr�sse 'tiny' f�r die Nummern.
	breaklines=true,         % Zeilen umbrechen wenn notwendig.
	breakautoindent=true,    % Nach dem Zeilenumbruch Zeile einr�cken.
	postbreak=\space,        % Bei Leerzeichen umbrechen.
	tabsize=2,               % Tabulatorgr�sse 2
	basicstyle=\ttfamily\footnotesize, % Nichtproportionale Schrift, klein f�r den Quellcode
	showspaces=false,        % Leerzeichen nicht anzeigen.
	showstringspaces=false,  % Leerzeichen auch in Strings ('') nicht anzeigen.
	extendedchars=true,      % Alle Zeichen vom Latin1 Zeichensatz anzeigen.
	backgroundcolor=\color{ListingBackground}} % Hintergrundfarbe des Quellcodes setzen.


\lstset{
  basicstyle=\small\ttfamily,
  columns=fullflexible,
  showstringspaces=false,
  %commentstyle=\color{gray}\upshape
}
%neue Lang definieren, als Bsp.
\lstdefinelanguage{XML-changed}
{
  basicstyle=\footnotesize\ttfamily\bfseries,
  morestring=[b]",
  morestring=[s]{>}{<},
  morecomment=[s]{<?}{?>},
  stringstyle=\color{black},
  identifierstyle=\color{darkblue},
  keywordstyle=\color{cyan},
  morekeywords={xmlns,version,type}% list your attributes here
}

%-----------Caption Package-------------------
\usepackage{caption}
\DeclareCaptionFont{white}{\color{white}}
\DeclareCaptionFormat{listing}{\colorbox[cmyk]{0.43, 0.35, 0.35,0.01}{\parbox{\textwidth}{\hspace{15pt}#1#2#3}}}

\DeclareCaptionFormat{graphics}{\colorbox[cmyk]{0.43, 0.35, 0.35,0.01}{\parbox{\textwidth}{\hspace{15pt}#1#2#3}}}


\captionsetup[lstlisting]{format=listing,labelfont=white,textfont=white, singlelinecheck=false, margin=0pt, font={footnotesize}}

%-----------Header+Footer---------------------------------------------------
\usepackage{fancyhdr}					%
\pagestyle{fancy}							%

\fancyhead{}
\fancyfoot{} 
\renewcommand{\headrulewidth}{0.4pt} % Kopzeilenlinie
\renewcommand{\footrulewidth}{0.0pt} % Fusszeilenlinie 0.0pt blendet sie aus

\renewcommand{\chaptermark}[1]{\markboth{\thechapter\quad#1}{}}
\renewcommand{\sectionmark}[1]{\markright{\thesection\quad#1}}

\fancyhead[LO]{\small\sffamily\rightmark}
\fancyhead[RO]{\small\sffamily\thepage}

%-----------Um die Eidesstattliche Erkl�rung als PDF einzubinden-----------------------------------
%\usepackage{pdfpages}

%------------Glossar--------------------------------------------------------------
\usepackage{expdlist}
\usepackage{glossar}

\renewcommand{\glshead}{\chapter*{Glossar}}
\renewcommand{\glentry}[2]{\glossary{#1@[#1] #2|glspage}}
\renewcommand{\glsgroup}[1]{{\listpart{\makebox[0pt][l]{\rule[-2pt]{\textwidth}{0.5pt}}{\textbf{\large #1}}}}}

\makeglossary

%Glossar mit Bordmitteln ------------------------------------------------------------------
%Darstellung des Glossars einstellen
%\usepackage[style=super, header=none, border=none, number=none, cols=2, toc=true]{glossary}
%\renewcommand{\glossaryname}{Glossar}
%\printglossary

% --- diverse Schriften -------------------------------------------------------------------
%\newcommand{\url}[1]{{\sf\small #1}}     % Hyperlinks


% -------F�r ToDo-Notes--------------------------------------------------------------------
\usepackage[color=red, shadow]{todonotes} % ", disable" deaktiviert ToDo-Notes
%Vereinfachtes "Inline-Todo"
\newcommand{\td}[1]{{\todo[inline]{#1}}}
\newcommand{\tdu}[1]{{\todo[inline, color=green!40]{#1}}}

%--------F�r Links-------------------------------------------------------------------------




%--------HyperRef konfigurieren-------------------------------------------------------------------------

\usepackage[
	pdftitle={\TitelArbeit},
	pdfauthor={\DeinName},
	pdfsubject={\TitelArbeit},
	pdfcreator={MiKTeX, LaTeX with hyperref and KOMA-Script auf Basis der Vorlage von seiler.it},
	pdfkeywords={Abschlussarbeit, Bochum, Hoschule},%weitere Keywords hier einf�gen
	pdfpagemode=UseOutlines,%                                  
	pdfdisplaydoctitle=true,%                                  
	pdflang=de%                                              
]{hyperref}

\hypersetup{%
	colorlinks=true,%        Aktivieren von farbigen Links im Dokument (keine Rahmen)
	linkcolor=LinkColor,%    Farbe festlegen.
	citecolor=LinkColor,%    Farbe festlegen.
	filecolor=LinkColor,%    Farbe festlegen.
	menucolor=LinkColor,%    Farbe festlegen.
	urlcolor=LinkColor,%     Farbe von URL's im Dokument.
	bookmarksnumbered=true%  �berschriftsnummerierung im PDF Inhalt anzeigen.
}
