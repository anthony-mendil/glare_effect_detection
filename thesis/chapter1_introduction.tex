\chapter{Introduction}
\label{introduction}



- papers von denen lesen. da sind richtig gute ineleitunegn an denen man sich orierntieren kann. \\
- 


- relevance of topic
- glare detectin has been done before with for example a light detector but not with behavioural data (volatile?) 
- 

Neural networks greatly benefit from lots of data. As the number of participants that provided data is very limited, more data can potentially improve the classification results.  The cognitive systems lab already implemented a system that takes the original game logs and simulates user behaviour in order to create new logs. As a result it is possible to create any number of games out of a single original one, from which some may be better than others. However, in order to simulate glare effect games multiple addtional steps and changes are neccesary. \\

- gereal approach, simulation, matrix creation, often used 1d cnn, novel approach of creating synthetic images and using 2d cnns shows in many cases improved classiiction resuts. 
Finally four voting based system one each for 5, 10, 15 and 20 rounds are implemented, combining the prediction of 400 models each to decide the final predictions.

- vielleict begrüdung: referenz zu denen. weil es zietgt dass man unter verwnedung simulierter spiele eine deutliche verberseerung der ergebnisse auf den echten spelen sehen kann...\\

- Einleitung: Was es für verschiedene Obstacles gibt, was schon gemacht wurde, was ich mache,

- one dimensional convolurional neural netweotks refered to as 1d convnets