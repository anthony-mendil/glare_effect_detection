\chapter{Memory Game}
- Bei dem zum simulieren des glare effects genutzen spiel handelt es sich um Memory (matching pairs game).  Das spiel enntält diverse Spielmodi, in welchen verschiedene memory oder visual obstacles existieren. Beispielsweise kann das Hinderniss einer allgemeinen sehschwäche, der rot grün schwäche oder auch, was für diese arbeit von bedeutung ist, das Blenden des Bildschirms durch einfallende sonnenstrahlung simuliert werden. Ebenso bietet das spiel mehrere möglichkeiten der adaption, wie besipielsweise das erneute vorzeigen der karten aus dem letzten zug oder die zuweisung von buchstaben zu kartenpaaren welche beim umdrehen einer karte ausgesproche werden. Erstere stellt einen gedächtnis assistenten für memory obstacles da während die zweite im fall visueller Hindernisse eine neue art erzeugt anhand der sich die spieler orientieren können. Dies sind nicht alle funktionalitäten die das spiel bietet, aber da nur bestimmte von relevanz für diese arbeit sind werden nicht alle erläutert. Für diese arbeit sind von besonderer bedeutung der modus ohne hindernisse und der modus mit glare effect. 



Das spiel wurde von mehrfach weiterentwickelt oder verändert. Ursprünglich bestand das spiel nur aus einen Tier karten set mit dem allgemeine Sehschwäche modelliert wurde. Später wurde es durch farbige Karten erwietert um auch spezille HIndernisse wie die Farbschwäche zu betrachten. Der stand ende septemer 2020 wird im folgenden erläutert, wobei sich nur auf die modi mit farbigen karten bezogen wird.

classis no obstacel: normales memory spiel

memory obstacle: memory spiel mit zusätzlicher neben aufgabe:  Bei jedem Kartenflip wirdeine zufällige Zahl zwischen 1 und 10 genannt, welche der Spieler konsekutiv aufaddieren muss. Nach Finden des letzten Paares wird der Spieler nach einer Summe gefragt, welche im Gamelog neben der exakten Summe gespeichert wird.

visual obstacle: zwei arten, color blindeness  oder glare effect (achtung verschiednen farben bei beiden!)

memory assistance: 

visual assistance: 

Jeder modus kann mit einem der asistenten oder ohne assistenz gespiet werde. (stimmt das? auch bei glare effect?)

Für diese arbeit sind nur von relevanz der classische modus ohne hindernisse und adaptionen und der visual obstacle glare effect modus ohne adaptionen. Abbildubegn re und ref zeigen die die beidem modi mit allen karten face up. 