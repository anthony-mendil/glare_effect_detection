\chapter{Conclusion}
85 \% etc.  

\section{Prospect}
- es wurde zwar impolementiert nicht alle features zu verwenden, aber es wurde nie ausprobiert. Hätte man testweise machn können. \\
-Vielleicht hätte man ein weiteres statistical feature für penalties machen können, so wie es beim qualitätscheck der simulationnen berechnet wird.\\
- statistical tests could be used to validate whether the changed to the simluator iproved the quality o the simuations. Howevr, as shown by the classification results even though they seemed to improve the quality or at least match it, the classification results were noticebly worse when usidm the simulated games ceaed by the modified simultor. There ae many variables and factors that influnece the final result, which makes it hard to validate whether changes to the simulator improved the overall results, even when using statistical tests. 
- a user study could be done to validate the voting based systems in real scenarios. As mentioned in chapter \todo{ref} the results shown in tabel \todo{ref zu table mit voting ergebnissen} 
- real world sacenario 20 modelle ohne splits, offline test leace one out, online test future work (alle data 20 repitions) (vielleicht wird es sogar minimal besser weil man die 20 modelle auf 20 statt auf 19 daten + simulierte ratio trainiert)

- man hätte ausgehend von intialen trainingsergebnissen noch mehr trainieren können in die richtigunen, aber wegen zeitgründen nicht gemacht 


- outlook in study: man fragt die leute ab wann sie etwas nicht 
mehr unterschieldoch wahrnehmen und berechnet dauraus eine neue bedeutung in der tabelle herleiten 

- mehrfach simulieren und immer nur beste nehmen. Aber: sehr zeitaufwendig