\chapter{Simulator}

Neural networks greatly benefit from lots of data. As there are only 40 \todo{richtige zahl herausfinden} overall games from participantzs available, more data can potentially improve the classification results.  The cognitive systems lab already implemented a system that takes the original game logs and simulates user behaviour in order to create new logs. As a result it is possible to create any number of games out of a single original one, from which some may be better than others. However, in order to simulate glare effect games multiple addtional steps and changes are neccesary. \todo{vielleicht so ein worklow diagramm erstellen..erstellung verebsserung etc}\\



- vielleich statt die beiden unter abscxhnitte infach nur beides in einem 
\subsection{Concept}
- generelle funktionsweise des simualtors
\subsection{Similarity Matrix}
- auch erklären was similarity matrix ist, struktur und wie die benutz wird\\
- and colour representation cie etc delta e color differnece, vs rgb was das bringt\\ 
- no special similarity matrxi for noosbst needed. The differneces can simply be calculated. it is already used in the simulator. However a new similarity matrix needs to be created for simulating glare effect games. 
- für no obst wird identity similariity matrix verwendet: diagonale ist 