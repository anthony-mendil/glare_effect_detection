\chapter{Voting based system}
The accuracies shown in section \todo{ref} are not representitive of how the models would be used in resal life scenarios. Since the accuracy is the mean accuracy of 20 repititions for each split there is no single modeel that produces that accuracy. For using the modles in real life scenarios a voting based system is deployed. It consists of multple models classifying the input data and then voting for the final classification. The accurcy of such a system more accurately describes the real world performance. Therefore a voting based sytsem is created each for the best cnns and ratios of simulated to real data for each number of step. As mentioned in section \todo{ref} the 1d cnn creates the best accuracy for 5 and 10 rounds while the 2d cnn creates the best accuracies for 15 and 20 rounds. Hence, in total 4 voting based systems are created, one for each of the 4 number of rounds. Each of the 20 repititions of a split is tested as before, only with the two real games (one no obstacle and one glare) that were not included in the training. However instead of direclty calculating the accuracy of each repitition of that split, the output labels are saved for all repitions and used to perform a voting for the final label. Meaning that if for instance 15 out of 20 repitions classyfied the game as a glare effect game, while the other 5 classified it as a no obstacle game, the final classification will be that of a glare effect game. This is done for the two test games. Using the final labels created by the voting, the accuracy is calculated. In the example above, if assumed that both final labels would be correct, the accuracy of that split would be 100\%. This procedure is repeated for each of the 20 splits and the accuracies of are averaged. 

.. table of accuracies of the 4 voting based systems ...

The voting based system produce noticeably increased accuracies. ... analyse... 


It would be possible to use these four systems during a game. Depending on the number of step a differet system is used. In reality when using these systems, the accuray produced by the voting could be even more improved. The reason for hat is that the voting cann be done by all 400 models instead of only 20 for each split. When testing the voting was done for each split indipendently because the other for each split being tested, the test data was used in the other 19 splits for training. This is not the case in real world scenarios when the game is being played. 